\subsection{Solution Description and Results}
[ Course A ] [ Course B ] [ Course C ] 
\newline
Describe the solution, test, and evaluation results.
\subsubsection{2 Factor Authentication}
First a verification model is created to ensure that the solution works logically. Once the solution is validated with the test, implementation can proceed. It is important to ensure that the system works as intended, especially when it comes to the security part of the system. The model is modelled the same way it works in the code described in the following sections.
\newline
\textbf{Logging into the system}\newline
A Bluetooth address is defined by 48bits [https://www.bluetooth.com/bluetooth-resources/intro-to-bluetooth-beacons-copy/]. Once someone makes a connection to the system, we gather all of these bits in a struct which is set for latest unit. Latest unit will be compared to a list of verified Bluetooth address', if any match it will return true. This is done every time before it goes into an unlock state and the door cannot be opened without a physical key.
\newline
\textbf{Adding device to authenticated}\newline
To lower the amount of verification checks performed, it will only perform a check after a connection has been established. If the button is pressed down during this period, the system will set the device as latest unit, and then iterate over each space in a list of verified devices, if the same device is already in there it will not add it to the list again, otherwise the new device will occupy the first free space in the list. The list has a max size, in this case it is 10, but could be altered for industrial preposes. Size is limited to not fill up space for potential future updates as once the system is in production, lowering the limit is problematic due to the possibility of removing otherwise existing users from authenticated list during an update.
\newline
\textbf{Verification}\newline
The testing of the systems works as intended with no unforeseen consequences, the system authenticates by comparing 10 times 48bits and the compute time here is negligible. The adding a new device to a system is done in: (FIND AND SET TIME!)

\subsubsection{Detection of Bruteforce}
This feature did not reach an implementation phase nor did it get verified with a UPPAAL model, it was in the end of analysis phase when the groups development stopped. In this section I will go over what and why. There is two major factors which caused this feature not to be finished, the first is the lack of function for it for Bluetooth classic[https://docs.espressif.com/projects/esp-idf/en/stable/esp32/api-reference/bluetooth/esp\_gap\_bt.html]. At the end of the development period (early May), it was not possible to get a list of the nearby Bluetooth devices and their ID through Bluetooth Classic GAP API. As our main product is build on this API, it was not possible as of yet to implement this feature without an update to the API. This feature exists for \gls{ble}[source], why it does this and not classic, is most likely due to classic being very slowly phased out, this belief stems from Bluetooth innovating solutions for Low Energy which classic is good at[https://www.bluetooth.com/blog/introducing-le-audio-bluetooth-le-audio-applications/]. We chose to work with classic to ensure that older phones would work with our lock. So the feature can only be developed on our \gls{ble} solution. That part of our system was finalized towards the end of, which did not leave enough room for developing an Low Energy solution only.

Operational it would work something like the following, be aware, as this feature was about to leave analysis phase, it was not yet modelled in UPPAAL. The system would make a count after advertising was finished, if there were too many devices it would disallow unlocking and to unlock, the user would either have to wait for a specified time or use a physical key. As the possible amount of bit combinations you can have is $2^{48}=2.8147*10^{14}$, then the amount of natural Bluetooth devices in an apartment complex would be lower than the max allowed devices nearby. If the limit was set to 1000 with wait of 5minutes between each attempt, then it would take an attacker up to $attempts*1000=2^{48} : attempts=56.294.995.342 :  1,4*10^{12} \frac{minutes}{attempt}$. Once the system has detected too many devices nearby it would flip a boolean, start a timer and forcefully lock the door. At this point it will not be allowed to make any other operation until the timer has reached the specified value. This way we lower power consumption due to not making any operations other than checking the state of the door in case of break-in.

As this feature was not finished, it is not possible to evaluate the solution and is put for future work.