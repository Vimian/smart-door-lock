\subsection{Problem Description}
[ SSaV ] [ MDSD ] [ IoT ] 
\newline
%Provide a more detailed problem description here. What is the problem about? Characterize the problem in a way that allows for deriving a solution.

%The problem is to implement a state machine in C code, so it can be run on the ESP32.
%The other problem is to interact with the Bluetooth hardware of the ESP32 through code.

There was given some requirements for the system which limits the scope of the design.
One of these being that the system must have a high level of trustworthiness, and must therefore act like the state machine in UPPAAL, which is proven to work by our UPPAAL tests.
UPPAAL does not allow for exporting the models to C code, so the state machine must be implemented manually.
Implementing a state machine in code does not give anything functionality by itself, an interpreter is needed to interpret the state machine.

Another requirement is that the system must be able to communicate with a cellphone, for which Bluetooth were chosen, in this individual report, the focus will be on implementing Bluetooth classic.
The Bluetooth hardware of the ESP32 is not easily accessible from the C code.
There has to be made some configurations and logic, to interact with the Bluetooth hardware.

To showcase the system a muck-up of the smart door lock will be made, which demonstrates the functionality of the system, using the available components.