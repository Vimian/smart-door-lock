\subsection{Problem Description}
[ IoT ] [ Course B ] [ Course C ] 
\newline
[ Phu ] [ Casper ] [ Author C ] 
\newline
%Provide a more detailed problem description here. What is the problem about? Characterize the problem in a way that allows for deriving a solution.
\begin{comment}
The IOT device in this case, ESP, is the chosen hardware used to connect the Smart door to the network it should support the Functional and non-functional requirements given in the case, The system is also based on SSAV course, that requires UPPAAL Diagrams to show the difference locations and atomic positions the state-machine can traverse. 
\begin{table}[]
    \begin{tabular}{lll}
    \textbf{Functional} & \textbf{Description}                                      & \textbf{MoSCoW} \\
    \textit{F1}         & The door should be able send a   command from a cellphone & M               \\
    \textit{F2}         & The door should unlock within 10   seconds or not at all  & M               \\
    \textit{F3}         & Faults of the smart door should be   detectable           & M              
    \end{tabular}
    \end{table}

    \begin{table}[]
        \begin{tabular}{lll}
        \textbf{Non-Functional} & \textbf{Description}          & \textbf{MoSCoW} \\
        \textit{NF1}            & Hardware has to be ESP Wrover & S               \\
        \textit{NF2}            & Bluetooth                     & S               \\
        \textit{NF3}            & WiFi                          & C              
        \end{tabular}
        \end{table}

These requirements as stated from the case Aswell as some features for us to implement such as through WiFi or Bluetooth these are some of the possible ways for ESP connectivity.
\end{comment}

%New thing
\subsubsection{Requirements}
Some requirements were given in the case, and from those, an idea will be chosen, which there later on will be added more requirements specifically tailored to the idea.
The requirements from the case are the following.
\newline

R1: Cellphone should be used to unlock the door.

R2: The door should unlock within 10 seconds or not at all.

R3: The system should have a high level of trustworthiness.

\subsubsection{Ideas}
From the requirements some solutions were arrived at mainly differing in the communication channel and ease of use for the user.
The requirements are the following.

Using Bluetooth as the communication channel.

Idea 1: The door should unlock when seeing a paired device nearby.
This requires no additional input from the user, you would just walk up to the door and once you reach the door, the smart lock would most likely already have unlocked the door.
Before a device can be used to open the door it needs to be paired with the ESP32, which allows for multiple devices to be able to unlock the same door.

Idea 2: The use of a mobile application, where the user would still approach the door like Idea 1, but then have to take out the phone, unlock the phone, find the application and send the unlock event, just to open the door.
The security for this idea would be similar to the first where only paired devices would be allowed to send events to the door.

Another approach is to use WiFi instead as the communication channel.

Idea 3: Using the ESP32 as an access point (AP), and then having the same flow as with idea 1 where the ESP32 detects trusted devices and unlocks and locks based on that.
The main difference from idea 1 is that the phone will auto connect to a WiFi which will conflict with home WiFi, meaning that when the user approaches the door if the phone connects to the ESP32’s AP the user will then have to manually which to the home WiFi afterwards to connect to the internet.
The other scenario is where the phone connects to the home WiFi the user will like in idea 2 have to take out the phone and perform some actions and connect back to the home WiFi.

Idea 4: Is about limiting the switch between Wifi networks and then getting the list of connected devices directly from the home router, which would require some way of getting the list of connected devices from the router.
This would have the same user flow as an idea 1.
Since it is a home that is being secured, additional any of the previous ideas could be extended with more points of verification, such as face recognition, fingerprint or others.

\subsubsection{Solution}
For this project idea 1 was chosen due to the ease of use for the user.
This idea is less secure compared to the others when it comes to bruteforce, as the intruder only needs to bruteforce one of the trusted devices Media Access Control (MAC) addresses.
Along with the requirement for securing this some other specifical requirements are added, these are the following.
\newline

R4: Allow pairing of multiple devices.

R5: 2FA for pairing devices. (pin and physical access to the ESP32)

R6: Allow for locking with a key while paired devices are nearby.
\newline

R4 is to allow multiple users in the same home, like a family.
R5 is to prevent bruteforce of the pairing pin.
R6 allows the user to lock the door etc. at night without having to turn off Bluetooth on the phone.